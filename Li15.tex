\documentclass{article}
\usepackage[utf8]{inputenc}
\usepackage{tipa}
\usepackage{qtree}
\usepackage{parskip}
\usepackage{amsfonts, amsmath, amssymb, amsthm}
\usepackage{graphicx}
\graphicspath{ {./images/} }


\title{Li15 First and Second Language Acquisition: Notes}
\author{Ashna Ahmad}


\begin{document}

\maketitle

\tableofcontents

\section{Introduction (first language acquisition)}
\begin{itemize}
    \item Stages of language acquisition:
    \begin{itemize}
        \item First stage: 0-2 (using senses to assimilate information; language that does exist is not social, but descriptive and used for labelling (used internally))
        \item 2-7: starts from single words, moves on to developing sentences, becomes symbolic and representational as well as descriptive, becomes able to discuss imagination, future, past etc.; animism (perceiving objects as animate) and egocentrism in language at this stage
        \item 7-11/adolescence: operational period - the shift from illogical to logical, during which the ability and affinity for abstraction is developed
        \item Even at the age of 9 months, children know a lot about language in terms of perception
        \item Language develops much faster than cognition; cognitive functions are at their peak from age 20-35 but solid native speech can develop around age 5 or 6.
    \end{itemize}
    \item Schools of thought in language acquisition:
    \begin{itemize}
        \item Constructivism: there exist cognitive precursors for language development. No innate linguistic categories and no aspects of language that are innate. Constructivism is based on the idea that the brain is in constant interaction with the environment - the more you can learn, the better you can analyse - and different levels of knowledge (e.g. between ages) impact the specifics of acquisition. Piaget: children do not think like adults; interactions with the environment condition them to do so over time as they construct their understanding of the world. Each stage of the development of logical thinking over time corresponds to different levels and responses to language. Objects having stable properties and other cognitive biases are accounted for by constructivism (/accepted); innate linguistic knowledge is not.
        \item Generativism/innatism: children acquire language easily because the child has inborn language faculty/some genetic endowment, which minimal environmental influence is required to activate. The old version of the Chomskyan approach involves a language acquisition device which is inborn, responsible for language development and includes UG. Principles encode similarities; parameters specify differences. The current version of generativism holds that UG only includes the merge operation which allows for syntactic tree construction, and requires language exposure (linguistic input) for activation.
        \item Generativism: language is innate, constructivism: ability to learn is innate, behaviourism: nothing is innate
    \end{itemize}
\subsection{Nativist vs. usage-based approaches}
    \item Broadly, nativist approaches to language acquisition are those which postulate an innate device or operation (e.g. MERGE) optimised for acquiring language and hold that environmental factors are insufficient to account for the rich and intricate language acquisition capabilities we find in humans. 
    \item Meanwhile, usage-based approaches stress the role of input and environmental stimuli, linking them directly to linguistic capacities; while they might hold that some innate cognitive abilities are required for language acquisition, they do not incorporate any inborn language-specific faculties into their account. 
    \item Nativist arguments often focus on the specification of overarching principles which are described as forming part of a Universal Grammar. These principles, it is argued, exist across linguistic boundaries and enable generalisation beyond what can reasonably be inferred from the input. 
    \item One example, originally stated by Chomsky (1975), is the ability to pick up structure dependenices. He cites the phenomenon of verb fronting in questions: by fronting the auxiliary \textit{immediately after the subject NP}, grammatical sentences such as “Is the woman who is singing [t] happy?” can be correctly formed. Crain and Nakayama (1987) find through elicitation tasks that this is indeed how children acquiring English form questions. Despite the fact that it would be easier to simply front the first auxiliary encountered (and not obviously wrong, since in simple, more common questions of the form  “is the woman singing?” this is what happens), not a single child in Crain and Nakayama’s study made this error.
    \item Childers and Tomasello (2001) studied children’s learning patterns when exposed to new transitive sentences (e.g. “the cow is pulling the car”) in five different training conditions (without verbal input, with familiar or unfamiliar verbs and with pronoun [“it”] or full [repeated] NP encoding). The children were tested on whether they could produce their own transitive sentences with pseudoverbs. They found that only the children exposed to transitive sentences with pronoun encoding (“The cow is pulling the car. See? He’s pulling it,”) were able to generalise the rule and apply it to their own productions. This study reinforces a key position in usage-based theory by showing, in contrast to nativism, that the children’s level of innate structural knowledge is poor and that acquisition is facilitated by certain modifications to the input experienced by the learners (i.e. the pronoun condition). 
    \item To sum up, nativists argue for strong innate structural knowledge and emphasise the need for a unified theory of the cross-linguistic origin of language acquisition aptitudes, while usage-based theorists argue for weak innate structural knowledge and emphasise the need to take into account lexical specificities and their grounding in experience in each language.
    \item Similarly, when nativists discuss the role of input, they argue that input does not account for the richness of linguistic knowledge underpinning acquisition processes (this is the well-known ‘poverty of the stimulus’ argument) and focus on demonstrating its gaps. Usage-based approaches, meanwhile, focus on establishing how exactly the input facilitates certain acquisition processes. 
    \item One illustration of the poverty of the stimulus argument from second language acquisition is the Overt Pronoun Constraint (Zyzik 2009; Hawkins 2008), which applies to null subject languages (e.g. Spanish) and states that overt pronouns and quantified expressions cannot corefer when the pronoun is in subject position. This constraint is not well-known to language instructors and is not taught explicitly. Hawkins argues that L1 English, L2 Spanish learners pick up this constraint with ease despite input which does contain examples of such coreference (overt, non-subject pronouns and quantified expressions can corefer) and despite the lack of an equivalent constraint in their native language. This is an example of how a nativist might deal with the role of input: by noting that there are subtle constraints which are picked up on by language learners even though they are not evident from input and not explicitly taught. Universal Grammar, therefore, fills the gaps and accounts for the subtleties which input cannot.
    \item An example of a usage-based account, meanwhile, would specify how the learner interacts with the input in order to construct a more descriptive theory, with the goal of building an account which is sufficiently comprehensive not to require any contribution from innate knowledge. 
    \item For instance, Zyzik (2006) describes how learners acquire the use of the clitic ‘se’ to indicate intransitivity in Spanish verbs which alternate between transitive and intransitive. Hypothesising that this rule is acquired on a case-by-case basis, she finds that errors involving the overgeneralisation of ‘se’ to transitive contexts are determined by the ‘chunking’ of the clitic with certain lexical items. This error is thus dependent on lexical input, and on how much exposure a learner has had to the sequences in which this lexical input normally appears. This sequence learning account is a method of specifying the direct role of input, which helps to provide more grounding to a usage-based description. We can thus see that regarding the role of input, nativists take care to emphasise the gaps left by input, while usage-based theorists specify its particular role in certain language acquisition processes on a case-by-case basis. 
    \item There are many more ways to specify the role of input. One common way is by examining the effect of perceptual experience (e.g. of spatial concepts such as distance and direction) on acquisition of the language used to describe this experience. Many usage-based theorists (cf. Goldberg and Caserhiser 2008) focus on how the ways in which we perceive objects’ physical relations to each other affect our grammaticality judgements (e.g. sentences of the form “She owned the metal flat” are never produced because we cannot physically visualise them, as opposed to sentences like “she hammered the metal flat”).
    \item Nativist theories ascribe a central role to syntax by enshrining many syntactic principles as a part of Universal Grammar, whereas usage-based approaches do not ascribe any more importance to syntax than to any other level of representation in language: for usage-based theorists, syntactic principles are acquired in an experience-driven, language-specific manner just like phonological and other rules. One syntactic principle argued by nativists to be universal is the Scope Principle, which states that c-command determines scope. In the sentence “I haven’t seen all my friends”, for example, the negative c-commands the quantifier and this determines that the negation extends across “all my friends.” 
    \item Usage-based theorists have used second language acquisition studies which question the universality of the Scope Principle in order to challenge nativism. If the Scope Principle holds universally, then L2 English speakers should have no problem applying it to English; however, O’Grady (2013) finds that this is not the case for South Korean learners of English. L1 English learners of Korean, however, do not have the same problem with scope rules in Korean; they are able to apply them correctly. O’Grady suggests that this is motivated by processing ease: learners transfer scope preferences from their L1 to their L2 if and only if this does not increase processing cost too much in the target language. This is an example of a usage-based, case-by-case rule which accounts for certain syntactic preferences lacking the universality which nativism ascribes to them. 
\paragraph{Conclusion} While nativist approaches concern themselves with finding a universal, innate set of principles which inform language acquisition and fill in the gaps left by input across languages, usage-based approaches are more concerned with examining the specific relations between input or experience and the process of language acquisition, on a case-by-case, language-specific basis. 
\end{itemize}
\section{Prelinguistic Communication}
    \subsection{Gestures}
    \paragraph{Gesture} A form of non verbal communication in which bodily actions communicate particular messages, either instead of speech or together and in parallel with speech (speech accompanying gestures)
    \begin{itemize}
        \item Types of gestures:
    \begin{itemize}
        \item Iconic: present concrete images/entities/actions
        \item Metaphoric: present abstract ideas/events
        \item Deictic: pointing
        \item Beats: flicks in time with the rhythm of speech
    \end{itemize}
    \item Adult pointing expresses:
    \begin{itemize}
        \item individual goals - the desired state of the world
        \item social intention (requestive, expressive, or informative) - what a person wants their conversational partner to know
        \item communicative intention - expressing the intention to communicate something; inviting the listener to work out the social intention. Not always deduced by children.
        \item referential intention - that the listener identifies a particular referent. This occurs relatively early on in children.
    \end{itemize}
    \item Referential pointing is normally iconic, and children will pick up on this early by following the direction of pointing when adults do it and observing how others respond.
    \item The earliest gestures are pointing and reaching, which initially occur without understanding of meaning. Towards the last quarter of their first year, children begin to pair pointing with eye movement in a particular direction (referential) and point to objects clearly out of reach to communicate the requestive social intention that someone else pass them the object. Eye tracking can thus demonstrate particular social and referential intent in children's pointing.
    \item Intentional pointing emerges in full around the first birthday. Infants rely on a joint attentional frame (mutual understanding) and an understanding of the intentions, attention and knowledge of the partner. In this sense, childhood pointing possesses theory of mind components shared by adult pointing; while pointing, children seek to establish common ground with others. 
    \item Although gestures are a pre-linguistic form of communication, it could certainly be argued that they are the first instance of a child transmitting communicative intent, and the first stage of their understanding of communicative rules.
    \item Other gestures in children:
    \begin{itemize}
        \item Iconic gestures used early on, and substitute for words in young children. Iconic gestures do not form whole `sentences' alone, but are frequently combined with speech to add additional meaning.
        \item Beat gestures are used when children start telling narratives
        \item Metaphoric gestures are used around 12 years of age
    \end{itemize}
    \item There is a high correlation between the onset of speech-gesture combinations and the onset of the `two-word stage' of speech (19-26 months). Children gradually progress, between the one-word and two-word stages, from gesture alone/with meaningless sounds to speech-accompanying gestures with semantic coherence and temporal synchronicity approaching adult levels.
    \item Speech-accompanying gesture allows them to express two elements of a proposition even when speech allows them to express only one. At the two-word stage, gesture stops expressing additional information and instead reinforces information about the speaker's intentions, just like gesture in adults.
    \item Gesture must be somewhat inherent - blind children gesture too and caregivers gesture less frequently when talking to children than when talking to adults.
    \item Compared the onset of symbolic use of signs and words in a group of 22 hearing children exposed by their parents to symbolic gestures from 11 mo old onward. Bimonthly interviews emphasizing contexts of use of gestures and vocal words indicated a small modality difference, thus providing support for the hypothesis that strides in cognitive abilities such as memory, categorization, and symbolization underlie this milestone in both modalities. However, the data also indicate that the small difference in onset time was reliable, thus providing support for the notion that the gestural modality is, in fact, easier for many infants to master once the requisite cognitive skills are in place. (Goodwyn and Acredolo 1993)
    \item The rate at which children use gestures at 14 months predicts the size of their vocabulary at 42 months. (Rowe et al 2008)
    \item When taught both novel words and novel gestures for objects, 18 month old children learn gestures and words, while 26 month old children learn only words - showing that by the end of the two-word stage children have learnt that words are the primary carrier of communication.
    \item In both spoken and signed language, for example ASL (Pettito, 1988), children in the process of acquiring language experience confusion between the signs or words for `I' and `you'. However, when the same signs are made as gestures, there is no confusion observed.
    \item Those who cannot physically gesture still learn language; once language arrives, gesture seems to become an optional accompaniment.
    \subsection{Babbling}
    \item ‘Babbling’ is the term used to refer to infant vocalisations , in any language, which typically begin around 6 to 7 months of age and consist of a subset of the child’s L1 inventory uttered with no apparent referential meaning. Mature language production differs from babbling in a few key ways: it is spoken with meaning, has a more complete sound inventory, and benefits from mature neurophysiology which imposes fewer constraints on the tongue movements and articulatory patterns available to the speaker. 
    \item Vocal development stages during the first year are predictable and universal: coos start between 2 and 3 momths, and babbling between 6 and 7 months.
    \item An articulatory filter influences input, rendering salient patterns with
    which child is familiar from own babbling.
    \item Overlap of babbled syllables and adult syllables give entry to adult
    speech system.
    \item ‘Good enough’ match of adult syllable and babble will make child use
    sound in relevant, frequent, and routine situations $\rightarrow$ small number of
    known lexical items produced in limited contexts
    \item The development of children's accuracy in reproducing their L1 phonetic inventory occurs in a U-shape:
    \begin{itemize}
        \item With more words, child starts to rely on temporarily activated
        representations for production, often impacting accuracy in word
        forms. Child’s previous production practise influences memory for
        word forms, and leads to holistic representation of words.
        \item Once child has learned some adult based words, word learning becomes easier and faster, as a result of emergence of well practised motor plans and templates serving to support attention and memory to form meaning link.
        \item Vihman et al: beginning of phonological systematicity in which child goes
        beyond individual word forms to develop patterns representing possible
        word shapes based on intersect between own output forms and common
        input forms
    \end{itemize}
    \item The non-referentiality of babbling makes it decidedly prelinguistic. abbling also has a much more restricted phonetic inventory than mature language – out of necessity, because infants lack the tongue speed and precision to realise all of the sounds present in mature language, and find it much easier to simply produce reduplicated syllables. Another argument for calling babbling a prelinguistic stage, rather than a stage of language production, is to highlight this physiological difference. By this account, children engage in the prelinguistic act of babbling while they lack the neurophysiological conditions to string together syllables and to produce a complete or near-complete phonetic inventory corresponding to a particular language. Once they are capable of producing referential, multisyllabic strings of sound which are capable of encompassing most of their L1’s phonetic inventory, their production can be referred to as ‘a stage of language production.’
    \item Locke (1986) found that across 15 different linguistic environments, including languages as varied as Afrikaans, Thai, Arabic and Latvian, the same consonants appeared in infants’ babbling. These included (usually dental and bilabial, and less commonly velar) stops and nasals, while fricatives rarely or never appeared (only Norwegian had more than one babbled fricative phoneme, while Spanish, German and Arabic had one each). An immediate explanation for this pattern is that these phonemes are quite simply easier for infants to pronounce: nasals and stops can be produced by raising and lowering the jaw. Similarly, Davis and MacNeilage (1995) analysed a corpus of transcribed English babbling and found that vowel-consonant combinations where tongue position is maintained – i.e. front vowels with front consonants and back vowels with back consonants – are significantly preferred over combinations which involve changing tongue position between the consonant and the vowel. Again, this has a simple explanation: maintaining tongue position makes the sound combination easier to realise. The strength of infants’ preference for sounds and sound combinations that are simpler to pronounce, across languages, shows that physiological factors are constraining sound patterns in babbling much more than they would in mature language. Based on this, one could argue that babbling is simply a case of infants pronouncing what they are able to - given their physiology - and learning to apply their mouth and tongue to produce sounds. This argument makes particular sense given the non-referential nature of babbling: instead of pronouncing certain words in order to convey referential meaning, as in mature language production, babbling infants make whatever sounds are physiologically easiest. This gives babbling a different, more physiologically-motivated nature to mature language production.
    \item Piaget claimed that that babbling is simply “a secondary circular reaction”: that is, a motor response which occurs as part of the broader perceptuo-motor development system and which does not result from the exercise of the language faculty in particular. Iverson et al (2006) sought to specify the relationship between babbling and motor skills as a whole by videotaping 26 infants as they played with rattles, before and after they had gained the ability to babble. They found that increased rhythmic arm activity – i.e. grasping and/or shaking of the rattle – was linked to babbling onset, and interpreted this as an indication that babbling as a skill is acquired and developed along the same timeline and with the same milestones as other rhythmic motor skills. While Iverson et al do not argue that babbling is unrelated to the language faculty, their study does point towards babbling being part of a broader motor skill set that is usually developed around the age of 6 to 7 months.
    \item Language-specific variation is very much present in babble, and supports the view that babbling is the first stage at which L1-specific production is acquired. For instance, there are rhythmic differences between French and English babble: the babble of L1 French babies shares the final syllable lengthening of mature French production, while L1 English babble has more closed syllables than French babble (Levitt and Aydelot Utman 1992). This shows that at the babbling stage, babies are already learning elements of their native language’s rhythm and phonetic structure: a step towards native language production. Another study showed that infants are perceptive to prosodic features of languages in their babbling phase – even when they have only had a few hours of exposure to the language. Sundara et al (2019) found that both bilingual English-Spanish babies and L1 English babies with five hours of exposure to Spanish alter the prosody of their babbling as a function of the language of the adult they are interacting with at a given time. When interacting with a Spanish speaker, both groups reliably produced more multisyllabic utterances and fewer closed syllables than with an English speaker, reflecting the higher proportion of multisyllabic words and word-final vowels in conversational Spanish. This suggests that while babbling, infants are sensitive to and are learning prosodic features – in fact, they are so sensitive to them that they only require a few hours of exposure to begin altering their babbling. 
    \item Bleile, Stark and McGowan (1993) looked at the speech development patterns of a child with otherwise normal developmental milestones who had received a long-term tracheostomy. The tracheostomy meant that the child was not able to babble during the normal babbling period, but did not affect any other areas of development. After the tracheostomy was removed, the child was monitored for a year, and their speech was delayed – leading to the conclusion that removing the ability to babble had slowed down the child’s acquisition of language. This would be explained by the idea that babbling is thr first stage of language production: if the first stage is delayed, naturally the remaining stages would be too. Of course, this is only a study of one child, but other research has found that earlier-onset and more complex babbling leads to larger vocabulary sizes later on (cf. Morgan and Wren, 2018). If the properties of an infant’s babble can directly predict the speed and quality of their eventual language acquisition outcomes, babbling must play a very direct role in the early stages of learning a language. 
    \item McCune and Vihman (2001) looked at the Vocal Motor Scheme (VMS) consonants produced by each of a group of children aged 9 to 14 months, defining VMS consonants as the specific consonant sounds which were produced sufficiently consistently by the children over the course of the experiment to be considered part of their inventory. They found that the number of VMS consonants a child could pronounce could predict at what age they started to produce words with referential intent, with the earliest speakers gaining the ability at 16 months and having been consistently able to articulate p and b sounds since the age of 9 months. This demonstrates a very close link between the acquisition of a phonetic inventory through babbling and the application of this phonetic inventory to referential lexical use. McCune and Vihman write that “the development of vocal motor schemes provides the critical minimum of differentiated phonological patterns needed to establish sound-pattern schemata appropriate to the verbal expression of meanings”; in other words, babbling is a first step towards establishing the sound-to-meaning mappings which are essential for language production.
    \paragraph{Conclusion} Babbling is non-referential, disproportionately includes certain phonetic combinations (nasals/stops rather than fricatives), and develops alongside motor development (Piaget) - but it also includes language-specific prosody, and the number of VMS consonants a child can produce between the ages of 9 and 14 months predicts at what age they can referentially produce words. (p/b since 9 months $\rightarrow$ consistent referential articulation at 16 months)
\end{itemize}
\section{Lexical acquisition}
\begin{itemize}
    \item Lexical entries are bundles of features including every level of representation.
    \paragraph{Phonological bootstrapping model} Children learn words by relying on phonological properties of the speech signal.
    \item This involves picking up on distributional regularities, word shape (coda/nucleus arrangement), and phonotactic constraints (whether a syllable is `allowed' in word-initial positions)
    \item Infants statistically learn patterns in auditory information - this happens whether said information is linguistic or nonlinguistic. On the basis of acoustic cues, infants build pre-lexical representations.
    \paragraph{Two-step model} Infants identify and store word forms by segmenting sound streams, and procede to link them to meaning. At 8 months, babies can store words without meaning; at 10-12 months, meaning-form association starts.
    \item Words have an extension (the entities they refer to) and an intension (the shared properties of the entities). Children pre-two-word stage often overextend the meanings of words to refer to anything with the same intension; this occurs in production only, not in comprehension.
    \item A vocabulary spurt begins at 20-24 months (mid 2w stage) and lasts until 6 years of age, with children acquiring 6-9 words per day.
    \item When vocabulary is between 50 and 200 words, sentence formation begins.
    \item The course of lexical acquisition:
    \begin{itemize}
        \item First words (up to 12 months): rate of development gradually increases from 12 months to about 10 years.
        \item Average vocabulary size of a 10y/o is 40,000 words; this increases to 60,000 by the age of 17.
        \item At 16 months, the bottom decile has a vocabulary size of 0-8 words, and the top decile has a vocabulary of 179-347 words, with an average of 40 overall. (Fenton 1994)
    \end{itemize}
    \item Children learn new words when they need them for communication (they recognise this need via joint attention), have the cognitive maturity to process the concept behind them, and sometimes (later on) when they have the grammatical representation to know where to slot them into a grammar.
    \item Some concrete nouns are learnt via association: by attending to an object which is being described, on multiple occasions, and thus picking up the word which is being used to refer to it. However, 30-50\% of the time, the child is not attending to the concrete stimulus.
    \item Cognitive biases specific to lexical learning:
    \begin{itemize}
        \item `Fast mapping' allows children to learn the gist of a word in context even after just one instance of exposure, and retain that information weeks later. (`Bring me the chromium tray, not the blue one.')
        \item Whole object bias: children map a new label to a whole object and not to its parts, substance or other properties.
        \item Taxonomic bias: labels are interpreted as referring to similar categories of objects (e.g. animals) rather than thematically related objects (e.g. a cow and a milk pail).
        \item Mutual exclusivity bias: children will conclude that a new label for an object that already has a label will refer to some part of the object or a property. This does not apply to bilingual children,
    \end{itemize}
    \item Nouns are universally acquired earlier than verbs (common nouns make up over 50\% of vocabulary for children with 100-200 word vocabularies) and action verbs (go, run) are acquired earlier than experiencer verbs (look, think). These patterns do not reflect input frequency, and are likelier to reflect a cognitive constraint.
    \item Why are nouns learned earlier than verbs?
    \begin{itemize}
        \item Verbs encode relational meaning: first we learn the objects, then the relations between them
        \item Relations are language-specific, because they reflect more abstract and perceptually-conditioned situations than common nouns (e.g. path vs manner verbs)
        \item If children learn object terms before relational terms, it follows that sufficient cognitive maturity to learn nouns is attained before sufficient linguistic maturity to understand how relational terms (verbs, prepositions) fit into the grammar and pragmatics
    \end{itemize}
    \item Cognitive maturity and level of experience (of certain phenomena) have a significant effect. High frequency of a certain word in input will not enable the child to pick up the word if the child lacks the maturity to understand the concept it refers to.
    \item Gleitman: extralinguistic information grounded on the nouns which have already been learned enables the relational concepts encoded by verbs and prepositions to be related to the real world. The problem is not insufficient linguistic faculty development to process relational concepts, but a lack of a means to relate these concepts to the real world in the absence of learned nouns.
    \item Human Simulation Paradigm: measuring adults' ability to derive meaning from various combinations of cues (function words without nouns, nouns without extralinguistic information, scenes in context without sounds). Function words (and therefore syntactic relations) were surprisingly useful cues for meaning.
    \item Concrete, imageable nouns are learned first through observation and in a a-syntactic manner
    \item Co-occurrence of concrete nouns with ‘mystery’ verbs increases learning efficiency.
\end{itemize}
\section{Time and Aspect}
\begin{itemize}
    \item Tense (past, present etc), aspect (perfective, imperfective etc) and Aktionsart (state [stationary] vs event [motion]) are components of time expression in language.
    \item The following Aktionsarten were identified by Vendler:
    \begin{itemize}
        \item States - durative, unbounded (unchanging)
        \item Activities - durative, unbounded, but in phrases
        \item Accomplishments - durative, bounded (inherent endpoint) (`walk to school')
        \item Achievements - punctual, bounded (inherent endpoint) (`reach the summit')
    \end{itemize}
    \item Time can be expressed by verbal inflections (tense), temporal adverbials (`for many years' etc) and temporal particles (somewhere between adverbials and suffixes)
    \item Relations between reference time and speech time correspond to the tense categories past/present/future
    \item The position of reference time relative to event time corresponds to aspect (anterior, posterior, simple)
    \item Time can also be expressed via discourse principles, e.g. the order in which we arrange items in a sentence. The Principle of Natural Order suggests that we order events in speech in the same order in which they occurred.
    \item It is difficult to separate children's internal understanding of order, simultaneity, synchronisation etc from the temporal expressions they may have picked up verbatim from adults, without understanding their temporal content. (form/function)
    \item When asked `which car drove the furthest?' Piaget found that children could not separate time and space (i.e. temporal order from spatial order, or duration from path traversed). The child may be reasoning that going more quickly necessarily covers more space, and covering more space requires more time.
    \item An operational understanding of time arrives when the child understands that time taken is the ratio of distance covered to speed. This requires understanding of reversibility of thought.
    \item Stages:
    \begin{itemize}
        \item Understanding of object permanence (an object exists even when not visible): 9-12 months
        \item Reversibility of events: 6-7 years
        \item Decentering (building temporal representations beyond the here and now): 7-8 years
        \item Operational understanding of time: 10-11 years
    \end{itemize}
    \item Acquisition of form is not necessarily acquisition of temporal function. If Piaget is right, we expect operational time as a prerequisite for correct linguistic expression of time (8-9 years of age)
    \item Order of acquisition of morphemes seems the same regardless of input (measured by defining obligatory contexts)
    \paragraph{Defective Tense Hypothesis} Following from Piaget's observation that children before 6 or 7 years of age cannot conceptualise tense relations, children before this age use tense markers to mark aspectual relations
    \item Katherine Nelson: cognitive ability to order events exists at 2/3 years of age
    \item Heike Behrens: longitudinal study of German 4 y/o child who can talk about the past and objects not physically present, earlier than Piaget suggested
    \item Conflation of tense and aspect: progressive -ing emerges first for activity verbs, past tense for (end point) achievements, and third person singular -s for verbs indicating placement of objects (early use of inflection seems to depend on verb semantics)
    \item An enactment paradigm study (reenacting events) showed that present tense was more often used to convey an imperfective sense, and the past tense for a perfective sense
    \item Children's acquisition of distinctions such as past vs. nonpast , perfective vs.
    imperfective, and bounded vs. unbounded may differ depending on the
    language being acquired: Weist et al. al.(1984), and Weist (1986) show that young
    Polish children do mark tense with all verb types.
    \item Discourse factors are more important determinants of verbal inflections: tense
    and aspect play a role in marking the background vs. foreground of discourse.
    \item When exactly do tense and aspect arrive in the language acquisition process?
    \item The telicity parameter encodes aspect independently of tense and is language-specific
\end{itemize}
\section{Early acquisition of syntax}
\begin{itemize}
    \item The acquisition of verb meaning is supported by the word-to-structure pairing procedre. So far we have seen thst extralinguistic information (observations from the environment) lead to the learning of concrete nouns before these nouns (along with nascent understanding of syntactic structure, and extralinguistic information) can be used to learn verb meanings.
    \item For syntax to develop, the child needs:
    \begin{itemize}
        \item Mapping noun phrases to event participants (argument structure)
        \item Early abstraction of syntactic frames
        \item Understandimg of discourse continuity
    \end{itemize}
    \item Syntactic features to acquire:
    \begin{itemize}
        \item Transitivity, intransitivity, ergativity, ditransitivity (i.e. subcategorisation frames)
        \item Differences in semantic role assignment (is the subject the agent or the experiencer?)
        \item Differences in aktionsart (stationary vs in motion)
        \item Difference in semantic/conceptual/cognitive features (verbs denoting emotions vs. verbs of action etc - similar to the agent/experiencer distinction)
    \end{itemize}
    \item Children map each noun phrase in a sentence to a participant role in the event encoded by the sentence (one-to-one). Word order is exploited very early on as a cue to which noun might be the agent and which noun the patient (it is reliably exploited around 15 to 17 months of age).
    \item Two-year-olds have a strong preference for transitive verbs to be associated with causative actions (which involve agent/patient relations).
    \item Usage-based account: children represent early experience with language in concrete terms, e.g. by counting the nouns and verbs in a sentence and deducing their role from that (so three nouns might hint at an indirect object). According to this account, children do not have early understanding of predicate-argument structure.
    \item Do syntactic frames (e.g. transitive-looking verbs, verbs which take sentential complements, verbs which take directions) cue meaning? It seems plausible that a child will learn to associate the first with a causative action, the second with a thinking/saying verb, and the third with a verb of motion.
    \item Blind children interpret `see' as `touch' and `look' (active) as `explore'; this is syntactically cued 
    \item Two-year-olds can distinguish between contact verbs (scratch, bite, pinch) and causative verbs (break, open)
    \item They can also link familiar semantic categories to novel verbs' argument slots through listening ecperience alone.
    \item Probabilistic bias towards discourse continuity; children can guess the role of a missing argument (e.g. an elided one) based on the immediately preceding discourse.
    \item Two-word stage: no clear grammatical structure or morphological patterning, but combinations of two words are of similar form and express similar meanings.
    \item Braine 1963: early utterances have a pivot grammar, where they have either two open words or one pivot and one open words (pivots include more, no, again, it; more includes common nouns, lemmas and adjectives)
    \item UG notions, such as semantic properties of language (attribute, object, action) are mapped onto universal syntactic categories and sentence structure. Based on those, language-specific grammar can develop (i.e. parameters). Remember the reliance of syntactic relation development on concrete nouns and their meanings.
    \item More stages of development:
    \begin{itemize}
        \item Two-word stage (telegraphic stage) at 18-24 months
        \item Inflectional categories, determiners, negatives and some prepositions at 24-36 months
        \item Interrogatives and subordinate clauses at 3-4 years
        \item Passives and conditionals after 4 years
    \end{itemize}
    \item Tomasello observed that 1-3 year old children produce only one type of verb construction, and may overgeneralise later.
    \item In the three-word stage, as soon as children produce full sentences they show correct word order patterns (an L1 SVO child will not start using VSO order) (early macroparameter setting?)
    \item How is word order acquired?
    \begin{itemize}
        \item Generative: the head directionality parameter is set, based on a small amount of trigger sentences, and language-specific word order patterns (parameters) are rapidly acquired
        \item Usage-based: individual verb patterns are observed and children generalise from them. This process requires a relatively large amount of input.
    \end{itemize}
    \paragraph{Verb-Island Hypothesis} Children learn the appropriate use of grammatical relations on a verb-by-verb basis
    \item Younger English-speaking children will reproduce VSO word order when exposed to it, indicating that they are closely attending to syntactic patterns in the input and reproducing it 
\end{itemize}
\section{Subjects, questions and relative clauses}
\begin{itemize}
    \item How much hierarchical structure is ingrained in children's early representations of syntax?
    \item Is acquisition driven by parameter-setting or does it proceed construction-by-construction?
    \item In what order are functional categories acquired? What triggers the development?
    \paragraph{The Continuity Hypothesis} UG principles are always available, and UG parameters (language-specific) are triggered by a small amount of exposure to input.
    \item Finiteness is a functional feature associated with subject-verb agreement, verb movement (e.g. V-to-C in V2 languages), subject realisation (as opposed to no subject e.g. in Spanish [null subject parameter])
    \item Finiteness is optionally marked in the two-word stage (Optional Infinitive).
    \item If functional categories are available from the first stage of language production, why is there an Optional Infinitive stage?
    \begin{itemize}
        \item Some functional categories are underspecified.
        \item Children have not acquired necessary discourse conditions.
        \item Constructivist answer: children are misanalysing input chunks.
    \end{itemize}
    \item Subject data:
    \begin{itemize}
        \item There is a delay in the emergence of plural morphemes (because contexts for plural agreement are initially absent?).
        \item 18- and 24-month olds are sensitive to grammatical and ungrammatical subject-verb agreement (which is a structural dependency)
        \item Children produce null subjects even in languages (like English) without the null subject parameter.
    \end{itemize}
    \item The emerging generalisation is that null subjects in child English are restricted to clause-initial positions in declarative sentences. They are usually found with infinitives, so finiteness errors tend to pattern with subject omission in finite clauses.
    \paragraph{Truncation Hypothesis} Children do not realise that CP is the root of all clauses (and this causes finiteness and subject omission errors)
    \item Children do not make structure dependence errors, but also do not tend to use complex questions often (which require understanding of structure dependence to be reproduced accurately)
    \item Generativists claim that this is evidence of an innate knowledge of structure dependence; constructivists claim that children simply observe that NP is a constituent and a relative clause is an NP containing a clause, and thus derive the correct formulation
    \item Children show wh-movement from early on, though sometimes not subject-aux inversion (T to C movement) within wh-questions
    \item Children have access to long-distance wh-questions from around 3.5 years of age. However, they sometimes produce ungrammatical medial wh-questions, which involve replacing the complementiser `that' introducing the subclause with another wh-element.
    \item It seems therefore that children are aware of intermediate C positions and cyclicity.
    \item Relative clauses appear around 3 years, however sometimes the subject DP is repeated again in the subclause (`the one that he lifted it').
    \item Most relative clauses produced by children are subject relatives, and children find it harder to understand long-distance relatives and wh-questions (e.g. with two wh-phrases) than producing them.
    \item Child productions show sensitivity to structural relations and language specific properties indicating a bias for structure and that children know which type of language they are acquiring or are sensitive to language specific input.
    \item Neverthless, there are systematic discrepencies with the input/target language which are challenging both for generative and usage-based approaches.
    \item Child production suggests that complex syntax is in place early (from 3 to 3.5years), but comprehension studies question this hypothesis.
    \item Complex syntax (e.g. object relative clauses) and deviations from the input like medial wh-questions cannot be straightforwardly accounted for as complex constructions abstracted from simpler schemata.
    \item To understand whether or how parameters guide acquisition a wider range of typologically variable languages needs to be considered.
\end{itemize}
\section{Passives and binding}
\begin{itemize}
    \item Early syntactic phenomena include basically everything in the above section.
    \item Late syntactic phenomena include:
    \begin{itemize}
        \item Movement and subordination (age 3)
        \item Focusing and topicalisation (age 3)
        \item Interrogatives (age 3)
        \item Passives (mastery at age 7)
    \end{itemize}
    \item Passives are rare in input, yet are produced spontaneously - albeit ungrammatically - after age 3.
    \item Passives of actional verbs are easier and earlier in acquisition than passives of `emotional' verbs - `was hit by' is easier than `was feared by'
    \item Children seem to be alright with A-movement from object to subject position overall; unaccusatives are easier to acquire than passives despite also involving such movement. There seems to be some particular, possibly cognitive barrier relating to emotional passives.
    \item At 34-40 months, abstract knowledge of passives is present; children are able to produce passives with novel verbs. Learning seems to be verb-specific.
    \item Passives appear to take longer because they require unlearning canonical word order mappings. The abstract syntactic representation of passives is available at age 6, but non-canonical mapping from arguments to thematic roles is not fully mastered until 9. (demonstrated by priming studies: priming occurs in both cases, but 6 year olds produce reversed passives)
    \item Sentence-picture matching tasks and priming tasks give different results regarding passive mastery - sentence repetition could be a more appropriate paradigm.
    \item Meanwhile, some binding phenomena (which are supposed to be part of the UG-endowed innate knowledge) are acquired fairly early. Children show adult-like comprehension and production of reflexives. (at 30 months old)
    \item However, pronouns (or rather matching pronouns to their referents) are more problematic; up to ages 5 or 6, children may interpret `John washed him' as `John washed himself.' This seems to be entirely a comprehension rather than a production problem.
    \item In other languages, where pronouns are clitics (e.g. lo/la in Spanish), pronoun acquisition is less problematic. So the morphosyntactic properties of pronouns seem to affect ease of acquisition/interpretation. Methodological differences do not affect the results of pronoun acquisition experiments in clitic languages.
    \item The delay in pronoun acquisition could be caused by processing difficulty, pragmatics (clitic languages have less varied coreference possibilities than English) and/or accidental coreference being confusing.
\end{itemize}
\section{Coreference and connectives}
\begin{itemize}
    \item To resolve anaphora, a child needs to be able to identify the anaphoric markers in the language, be sensitive to discourse constraints (topicalisation, shared knowledge) and have sufficient cognitive resources (working memory). Anaphora is a syntax-discourse interface phenomenon, only acquired after the child already has some notion of discourse constraints.
    \item In adult speech, speakers try to make referring expressions maximally interpretable to addressees, so they use pronouns for disambiguation even though NPs might be more informative.
    \item Children, however, cannot keep track of speaker's referential knowledge until age 5 or later, so referential expressions become adult-like around age 7. This is because young children cannot grasp the notion that speaker and hearer assumptions are always independent.
    \item The development of the ability to construct and tell a narrative is related to the development of character reference. Developing character reference requires sufficient linguistic, cognitive, and theory of mind capabilities.
    \item Each language may use different devices to mark information status (articles, determiners, case etc) so children acquiring different languages are faced with different linguistic ability constraints.
    \item Cognitive ability constraints, meanwhile, may involve memory retention, attention to different characters and events, and grasping the difference between new and given information.
    \item Definite articles seem to be acquired earlier. Younger children also seem to assume a greater degree of common ground between listener and speaker: they use pronouns more often than NPs (4-year-olds almost always use pronouns), and are more likely to omit arguments in situations of joint attention or previous mentions based on overextending common ground.
    \item At 7-10 years, parity with adults is reached.
    \item In narratives, young children use the thematic subject strategy (always pronominalising the main character) while older children pronominalise the previously introduced character and nominalise the newly introduced one.
    \item For reintroduction of a previously (but not immediately previously) mentioned character, use of more definite forms as opposed to pronouns increases between 4 and 10 years of age.
    \item Experimental conditions can be influenced by the nature of the question asked, as well as (for children over 2) discursive factors such as previous mentions and perceptual availability of the referent
    \item Young children (2 to 3) can produce appropriate referential expressions in experimental contexts, but there is significant delay in acquiring appropriate use of such expressions in extended narratives.
    \item The primary reason for delay seems to be sociocognitive: young children overestimate the interlocutor's knowledge and so overuse pronouns until they have a sufficiently good understanding of the interlocutor's perspective
    \item Meanwhile - the first connective children learn is `and', and they learn the additive sense followed by the causal, temporal, and finally adversative sense.
    \item In general, conjunction is learnt first, then complementation, then relativisation.
    \item Even 3 year olds can express causal relations through the use of connectives. 
    \item Relative late acquisition of `while' in a simultaneous sense (7 years of age)
    \item There are some crosslinguistic differences: English children acquire connectives faster than Thai children (via expressing temporality through tense and not aspect?). The encoding of temporal relations in different syntactic categories affects the order and speed of acquisition.
    \item Then, which signals the simple sequence of two events and together, which signals the co-occurrence of two events without specific reference to duration, are conceptually simple and are acquired early
    \item Since, which indicates both order and duration, is conceptually complex, and is acquired late (cross-linguistically)
    \item Sequentiality as encoded by then, before and after is acquired prior to simultaneity as encoded by while
\end{itemize}
\section{Introduction (second language acquisition)}
\begin{itemize}
    \item Errors reflect development trajectories, e.g. acquisition of the negation in English occurs in stages:
    \begin{enumerate}
        \item Negative marker (`no') at the beginning or end of the sentence
        \item Negative marker with verb-unanalysed `don't' (`don't' is not declined and precedes modals, ungrammatically)
        \item Negation on the auxiliary/modal instead of through `no' and `don't'
    \end{enumerate}
    \item Learners produce erroneous and correct forms interchangeably for a period of time.
    \item Development follows the same order in naturalistic and classroom learners, e.g. acquiring subject relative clauses before object relative clauses before relative clauses of obliques/possessors
    \item Fossilisation: cessation of language learning despite continuous rich input and language use. Fossilisation represents the significant difficulty of convergence to L1-like outcomes
    \item L1 influence on L2 development:
    \begin{itemize}
        \item Positive transfer: concepts shared between the L1 and L2 are acquired quickly
        \item Features absent from the L1 are challenging because the direct transfer approach which is adopted at the start of the SLA process is not favourable to them.
        \item Sometimes such features are avoided altogether by L2 speakers due to lack of confidence 
    \end{itemize}
    \item The linguistic distance from the target language (across all levels of representation) can predict scores in proficiency tests
    \item L2 knowledge can be obtained implicitly and subconsciously like L1 knowledge or it can involve conscious learning of rules and forms (learners who produce errors can correct 95\% of them)
    \paragraph{Acquisition vs learning hypothesis (debated but supported)} Conscious learning cannot lead to acquisition; learners will fail to acquire some extensively taught features without organic exposure.
    \item Krahsen: classroom-based acquisition can take place if learners are exposed to comprehensible input.
    \item Key empirical facts that any SLA theory needs to account for:
    \begin{itemize}
        \item The existence of systematic developmental stages of learner grammar
        \item The variability in learner production and coexistence of correct/erroneous forms in the same sentence
        \item The lack of a guarantee of successful acquisition (and high variance of L2 proficiency between learners)
        \item The pervasive impact of L1
    \end{itemize}
\end{itemize}
\section{Theoretical approaches to SLA}
\begin{itemize}
\item What is difficult and easy? Which aspects of grammar are susceptible to fossilisation? What is the impact of memory, age, and other constraints?
\item An L2 learner - unlike an L1 learner - has metalinguistic awareness and can be a conscious learner, filtering their output through the rules which they have consciously acquired
\item Generativists: explicit knowledge cannot become implicit knowledge.
\item Usage-based: drawing attention to rules and grammatical aspects of L2 assists implicit learning/knowledge
\begin{itemize}
    \item Skills theory: through automatisation, explicit knowledge can become implicit
\end{itemize}
\item Theoretical approaches:
\begin{itemize}
    \item Functionalism: eaning for communicative purposes drives learning of structure (function needs drive form acquisition)
    \item Cognitive/usage based approaches: general cognitive mechanisms enable learning from input and through use
    \item Generative approaches: L2 usage availability is debated
\end{itemize}
\subsection{Functionalism}
    \item First developmental stage: the learner knows unconnected nouns, adverbs and particles and is missing argument structure 
    \item Second stage: mastered by all learners given enough time, and involves discourse organisation, temporality and agent/non-agent distinction (known as the basic variety; no tense morphology, and context and sequence of discourse structure provides all temporal information)
    \item Third stage: incorporates functional morphology e.g. finiteness
    \item Syntactic development in L2 is driven by communicative needs and fossilisation occurs when improving mastery of syntactic structures no longer correlates with communicative advantage
    \item Variability in progression is caused by variability in motivation and needs 
    \item Commonalities and universalities in syntactic development are a result of universal communicative needs shared by all learners
\subsection{Usage-based approaches}
    \item `Thinking for speaking': learners need to acquire not just a new mapping, but a new concept (e.g. the imperfect/perfect distinction, which is a linguistic rather than cognitive notion - Slobin makes this distinction)
    \item Learners transfer the way concepts are realised from their L1, e.g. the translation of the present tense concept into continuous and simple sub-concepts
    \item L2 learners have the same general learning mechanisms as L1 learners.
    \item Construction-building proceeds from item-specific constructions to more general constructions, conditioned by frequency in input
    \item Learners pay attention to specific features which are either directly transferable from L1 or are particularly common (or significant to meaning) in input
    \item Overshadowing can prevent learners from attending to - and therefore from learning - particular features in input. These are normally morphemes and features that are absent from L1 and do not add anything to the meaning of the particular sentence.
    \item Repeated exposure and explicit instruction can prevent overshadowing and help learners to `implicitly' assimilate the concept.
\subsection{Generativism}
    \item ???
    \item UG could be difficult to reconcile with the rarity of native-like attainment. 
    \item L2 learning engages general inductive mechanisms: it is possible for learners to pick up an artificial language which violates UG constraints.
\end{itemize}
\section{Coreference in SLA}
\begin{itemize}
    \item `The approach to language study concerned with the functions performed by language (reference and meaning)'
    \item Discourse is only fully functional if person, time, space, event realisation/perspective and mood are all covered. Different languages may express these elements in different ways, but fundamentally, communicating in every language must fulfil these discourse purposes.
    \item Due to theory of mind, and the consequent understanding of which information is common to the speaker and the listener and which is not, all languages use fuller forms to mark new information and leaner forms for previously mentioned information
    \item New information tends to be postverbal and given information preverbal - across languages
    \item Disambiguating referents requires keeping track of gender, understanding degrees of coreferentiality and simultaneously tracking referents while also maintaining a sense of discourse continuity
    \item Cognitive prerequisities (world knowledge, shared knowledge aka theory of mind, given/new contract) and linguistic prerequisites (knowledge of form)
    \item (Hickmann 2003) In L1 reference maintenance, coreference is the main factor influencing the form of \textit{given information} (followed by subjecthood and agency). L2 learners, meanwhile, are overly explicit in subsequent markings of new information - this is almost a hypercorrection in comparison to L1 results (children will underuse `full' forms, adult learners will underuse `lean' forms)
    \item L2 learners easily grasp appropriate use of local markings for new information in discourse - but acquisition of global markings is language-dependent (e.g. learners of some languages recognise the link between new information and postverbal position earlier than learners of other languages).
    \item On the other hand, L2 learners take longer to acquire markers which suggest first mention (e.g. preverbal position) and produce incorrect subsequent mention forms, either by overexplicitness or a standard grammatical error like article omission.
\end{itemize}
\section{Morphology in SLA}
\begin{itemize}
    \item Recall `conflation of tense and aspect' in section 3, and the defective tense hypothesis
    \item Andersen 1989: early inflections are predominantly guided by aspectual characteristics of verbs, but this does not reflect on children's cognitive understanding of temporality
    \item Andersen 1990: there is not an equal distribution of possible inflections for each verb type. Activity verbs may well pattern more with progressive tenses, for instance.
    \item In some languages, certain aspect/tense combinations sound more blatantly wrong than others.
    \item L2 learners seem to make similar aspect/tense generalisations to L1 learners, associating past morphology strongly with achievement/accomplishment (perfective) and the progressive tenses with activity (imperfective). However, they associate state verbs with the progressive tense to, which is not found in L1 learners and may be an instance of direct transfer (??)
    \item The past perfective emerges before the past imperfective, and the past tense is more easily acquired in the context of achievement/accomplishment verbs. The past imperfective is more easily acquired with state/activity verbs.
    \paragraph{Distributional Bias Hypothesis} Both L1 and L2 learners are incorrectly inferring that the correlation between perfect tense and accomplishment (and between imperfect tense and activity) denotes an exclusive relation.
    \item This may be because:
    \begin{itemize}
        \item Aspect is more relevant to the meaning of the verb, so the tense marker takes on aspectual meaning (i.e. the meaning of the aspect which they are more strongly correlated with)
        \item Learners choose the morpheme whose aspectual meaning is more congruent with the aspectual meaning of the verb.
        \item Learners assume that each new verb meaning has one and only one meaning and function (e.g. progressive signals ongoing activity to the exclusion of the past sense) 
        \item Learners select the most prototypical meaning of each inflection and associate it with the most prototypical members of each semantic aspect class of verbs. This distinction is very strong early in the learning process. 
    \end{itemize}
    \item Similar distributional patterns exist in native speakers because congruence is still a concern for them (both while learning and, once they have reached proficiency, producing the intended effect of a sentence).
    \item Learners have to learn to decouple inflections from prototypical verb choices in order to express more varied perspectives.
    \item All of these processes can only occur when the learner is using inflections systematically.
    \item There are other ways to temporally anchor events, primarily via discourse order (principle of natural order) and connectives.
    \item Stages of temporality expression:
    \begin{enumerate}
        \item The basic lexical stage (only via lexical temporal markings)
        \item The free morpheme stage (free auxiliary-like morphemes used to mark perfect/progressive aspect and past/present tense)
        \item The packaging stage (the previously free morphemes acquire finite forms, and are now auxiliaries proper)
    \end{enumerate}
\end{itemize}
\section{Acquisition, learning and L2 learnability}
\begin{itemize}
    \item Is spontaneous production evidence of language acquisition? Can explicit/metalinguistic knowledge `automatise' and become implicit?
    \item If learners can achieve 90-95\% accuracy learning a morpheme, we can call it learnable - but in practice, performance is often a lot lower than this, hence a need to determine whether a morpheme has `theoretical learnability' value separate from observed outcomes.
    \item Crosslinguistic variation is encoded as variation in the formal features associated with functional elements, e.g. the element `the' is associated with the definiteness feature.
    \item Uninterpretable features require learners to pay attention to agreement and are language-specific. Can L2 learners access them if they are absent from the L1?
    \item Ionin (2013): third person -s and past tense -ed are more problematic morphological items (highest rate of correct production is 79\% - are these `unlearnable'?) in acquisition than auxiliary be forms. Errors of omission are more common than misuse of inflectional forms. Correct verb placement is common.
    \item The problem lies with mapping the formal/syntactic features to the surface forms: an issue with morpheme retrieval/performance rather than knowledge/competence.
    \item If be-forms are easily-acquired and verbs are placed correctly, it seems that T features are being picked up. The problem is not finiteness/T, but the mapping of specific functional heads to affixal morphology.
    \item If the only problem is retrieval of the correct forms during production, we would expect to see evidence of a correct underlying representation/correct underlying knowledge. However, comprehension tasks can be problematic too. Hence, McCarthy (2008):
    \paragraph{The Morphological Underspecification Hypothesis} The morphemes employed as defaults are always underspecified
    (or less specified), so less specified/analysed/inflected morphemes are more likely to be used (and overused) in production (e.g. the copular auxiliary?).
    \paragraph{The Representational Deficit Hypothesis} L2 learners cannot acquire [variously formal/uninterpretable] features absent from their L1. (Alternatively, these features are associated with morphology which is harder to acquire; this is known as the Interpretability Hypothesis.)
    \item E.g. Greek learners of English fail to acquire uninterpretable Case and Agreement features, while they use interpretable animacy features to guide their learning (erroneously)
    \item To compensate, learners can overrely on interpretable features, or resort to inductive/associative learning (i.e. highly explicit learning)
    \item Jiang (2004): unlike native controls, L2 learners do not slow down when processing agreement violations. This may mean that the relevant features are not present in underlying mental representations, else unexpected features would cause processing to momentarily slow.
    \item VanPatten et al (2012): native speakers and L2 learners of Spanish show sensitivity to subject-verb inversion and adverb placement, but only native speakers are sensitive to agreement errors.
    \item Learners therefore appear to have reset the verb movement parameter early on.
    \item Syntactic representations appear to be in place before morphological ones (though maybe this results from the study focusing on English learners of Spanish). Difficulties with inflectional morphology can be viewed as a performance issue or formal features in the underlying syntactic representations.
\end{itemize}
\section{Aspect and reference}
\begin{itemize}
    \item English has the [+perf] feature on verbs, and so bare/default verb forms such as the infinitive denote a completed event, and the addition of the progressive affix is required for an ongoing interpretation. Do learners acquire all aspects related to this feature at once?
    \item The acquisition of all [+perf]-related properties seems to occur in tandem - confirming that the featural approach is accurate 
    \item Past $>$ past progressive $>$ present perfect $>$ past perfect in order of acquisition (in English)
    \item Recall conflation of tense and aspect
    \paragraph{The Aspect Hypothesis} The acquisition of grammatical aspect is driven by the inherent lexical aspect of verbs.
    \item Narrative structure influences the acquisition of grammatical aspect (in addition to lexical aspect): in English, foreground is associated with past and background with progressive, and so picking up on this association in narrative structure has an impact
    \item `I propose that the grammaticized categories that are most
    susceptible to SL [source language] influence have something
    important in common: they cannot be experienced directly in
    our perceptual, sensorimotor, and practical dealings with the
    world.' (Slobin) 
    \item E.g. L1 influence on the acquisition of number is weak (Murakami and Alexopoulou)
    \item [+def] impacts context, e.g. if a nominal is easily recoverable from context it is less likely to include an article (the definiteness feature therefore participates in the syntax-discourse interface) and also impacts the properties of the noun (an article converts a predicative noun to an argument)
    \item Determiners as a syntactic head are always available - but other properties of determiners are subject to language-specific variation:
    \begin{itemize}
        \item In NP languages, conversion to an argument can occur via covert, `article-like' operators
        \item [+def] is more widely used in Romance languages (specifically, to refer to kinds)
    \end{itemize}
    \item All generative accounts assume universal availability of the definiteness feature; however, it is necessary to account for language-specific acquisition of the semantic and anaphoric uses of determiners.
    \item The difficulty of definiteness-marker acquisition comes from the complexity of the interface mappings (syntax/semantics and syntax/discourse)
    \item Order of definite article acquisition in English is immediate situation use > coreference use > location/situation-specific uses (Liu and Gleason 2002)
    \item Trenkic (2008): the indefinite article is overused in contexts where the speaker does not know the identity of the referent, showing that determiners/definiteness markers are acquired as an optional identifiability marker rather than an obligatory syntactic head
    \item Lardiere (2009): Spanish learners need to `reassemble features', i.e. dissociate the definiteness syntactic feature from the semantic feature marking a \textit{kind} of object.
    \item Belletti et al. (2007) shows that direct transfer of coreference-related features is a problem even at advanced levels (ITA/EN)
    \paragraph{Interface Hypothesis} Anaphora are harder to acquire because they involve external interfaces (rather than simply grammar-internal interfaces, like the syntax/LF interface)
\end{itemize}
\section{Bilingualism}
\begin{itemize}
    \item Balanced bilingualism is rare - most bilinguals are dominant in L1 or L2, where `dominance' is not necessarily linked to absolute proficiency.
    \item `Bilinguals are not two monplinguals in one person' (Grosjean) - due to processing differences, it is unwise to compare the two.
    \item Criteria affecting bilingualism:
    \begin{itemize}
        \item Age of onset (simultaneous vs late successive (variously after age 4 or after age 7) vs early successive bilinguals)
        \item Minatenance throughout lifespan
        \item Exposure to each language (quantity and quality/situation)
        \item Cognitive abilities (working memory, cognitive control)
    \end{itemize}
    \item A bilingual child's combined lexicon in both languages is similar to that of a monolingual child (with each individual language being smaller) (Vermeer 2001). Vocab breadth/depth are strongly related to that of their input. Perception of phonological segments and prosodic features are marginally delayed (Fennell, Polka and Werker 2002)
    \item Bilinguals have lexical retrieval problems (due to the lexical competition which causes increased processing loads) which lower verbal fluency (Bialystok et al 2011)
    \item Dernie 2001: in school age children, not speaking their language of education at home can be problematic for educational outcomes.
    \item Heritage speakers encounter L2 between ages 4 and 5, and L2 interferes to such an extent that L1 acquisition is left incomplete/that heritage speakers diverge from the trajectory of other monolinguals in their L1 around the age of L2 onset. Incomplete acquisition is usually greater in simultaneous bilinguals and early sequential bilinguals (i.e. when multiple languages are introduced at an early age).
    \item The correct acquisition of gender and null subject parameters is particularly interrupted by the onset of an L2 (as well as any other phenomena acquired relatively late). Inflected infinitives in Brazilian Portuguese are one area of which heritage speakers have demonstrably less mastery; this is attributed to the heritage speakers lacking exposure to Portuguese in educational settings (Pires and Rothman 2009).
    \item This interruption could be due to the erosion of previously-acquired linguistic properties (non-pathological attrition, e.g. Polinsky 2011) or incomplete acquisition (Montrul 2011)
    \item Without longitudinal data, it is difficult to distinguish between incomplete acquisition and attrition. Primary linguistic data (a `before' picture) is required to evidence attrition.
\end{itemize}
\section{Attrition}
\begin{itemize}
    \item Attrition is permanent and affects linguistic representations, unlike transfer effects.
    \subsection{Hypotheses concerning attrition}
    \item Regression hypothesis: attrition is gradual `unacquisition' (Jordens, de Bot and Trapman 1989)
    \item Threshold hypothesis: infrequently used L1 elements do not meet the activation threshold and representations are `deactivated' - however, frequency effects are mostly linked to declarative memory and therefore should not affect the underlying linguistic representations which are implicitly acquired.
    \item Activation hypothesis (Paradis 2004): as above, domain-specific L2 attrition is based on infrequent use, however infrequent representations are not deactivated per se, but are instead replaced by L2 representations in a domain-specific manner. In other words, if a speaker discusses elements in a certain area more often in their L2 than their L1, their more used (more activated) L2 elements gradually replace their less used L1 equivalents.
    \item Interference hypothesis: L2 influences L1 knowledge/processing
    \item Simplification hypothesis: morphology is the first area to lose representational fineness when a language is underused, for the purpose of Simplification
    \item Markedness hypothesis (UG): marked parametric values are lost (The concept of markedness has traditionally been invoked to try
        to provide a description and explanation of aspects of language that
        are are felt to be 'unnatural', infrequent, complex or lacking gener-
        ality. In the field of first language acquisition, it is usually
        assumed that marked forms or structures, however defined, are harder
        to learn and are acquired after unmarked)
    \item Dormant language hypothesis: the language is not lost as such, but simply harder to access. As a result, it takes less time to relearn a language than to learn it for the first time.
    \subsection{Approaches to attrition}
    \item Generative: interface hypothesis, i.e. syntax-discourse interface phenomena are more affected by attrition than syntax-internal ones 
    \item Usage-based: degree of attrition is based on frequency (of the given feature in L1 input) and cue reliability. Attrition is due to increased competition between L1 and L2 which weakens the L1 connections/activation.
    \item The ideas of ‘cue validity’ and ‘cue strength’, (i.e. how predictably a certain syntactic, semantic or phonological property of a word can deliver particular information about surrounding words and grammaticality) represent the information which is learnt from exposure to input during the process of language-learning.
    \item Schmid 2007: the amount of use of the L1 in daily life does not seem to have any predictive power for the attrition effect.
    \item L1 attrition begins early? There are early negative effects of L2 development on L1 lexical processing and access, voice onset time, and understanding of non-literal language (Baus, Costa and Carreiras 2013)
    \item Attrition effects can disappear following exposure to the L1, suggesting that attrition is a performance rather than a competence problem. However, if attrition affects underlying representations, it must have some competence component too (and evidence for the Interface Hypothesis - i.e. the notion that syntax-discourse interface phenomena are more easily attrited than syntax-semantics - points to this conclusion).
    \item Relearning an L1 is easier than learning it for the first time. Children exposed to an L1 in early childhood who have since lost all ability can learn it much faster than controls. This suggests that there is simply difficulty accessing the L1 representations which were built, rather than complete loss of these representations. This is consistent with the fact that input frequency effects mostly apply to declarative memory: underlying representations exist in procedural memory and are not eliminated from procedural memory by the lack of input. Instead, the memory itself is simply harder to access.
    \item However, event-related brain imaging data shows complete erasure of L1 knowledge in early adoptees.
    \item Critical age effect in attrition: children who lost L1 input under the age of 11 will experience much more attrition. The same applies to lack of exposure to a second language; an L2 can be attrited for the same reasons as an L1 (except that initial L2 proficiency is also a factor) and is subject to the same age effects
\end{itemize}
\section{Appendix}
\begin{itemize}
    \subsection{The Competition Model}
    \item MacWhinney and Bates’s Competition Model is supported by this dataset, as it seems to reflect an L2 learning process which is \textbf{heavily influenced by direct transfer from the L1.} The main tenets of the Competition Model are that \textbf{communicative function} is the main influence on language development (so language production is a process of translating function to form, while comprehension involves translating observed form into function in order to extract meaning) and that syntactic patterns are controlled by \textbf{interactions between lexical items within a network which fulfil complementary roles}, while semantic and phonological properties are \textbf{controlled by competing cues in these networks.}.
    \item The theory of item-based patterns [28] is based on ideas about positional patterns from Braine (1962, 1976)[32][33] that were modified by MacWhinney to restrict early patterns to individual lexical items. The theory holds that children learn item-based patterns to combine words by focusing on individual operators, such as "my" or "more". They learn these patterns as a part of their learning about the function of these operator words. They then generalize groups of item-based patterns to abstract feature-based patterns such as the one that places adjectives before nouns in English. In this way, the theory of item-based patterns can be viewed as an early instantiation of Construction Grammar,[34] albeit one specifically designed to account for child language learning. Sentence processing involves the repeated filling (or merge[35] or unification[36]) of roles specified by the item-based and feature-based patterns. 
    \item The linked dependency structure provide by the operation of item-based and feature-based patterns serves as the input to mental model construction. On this level, sentences encode agency, causation, reference, and space-time through a process of perspective taking.[29] This process allows the human mind to construct an ongoing cognitive simulation of the meaning of an utterance coded in linguistic abstractions, through the use of perceptual realities derived from one's embodied experience.[30] The perspective taking approach views the forms of grammar as emerging from repeated acts of perspective taking and perspective switching during online language comprehension
    \item Complexity arises from the hierarchical recombination of small parts into larger structures (Simon 1962). 
    \item The fully emergentist version of the UCM emphasizes the ways in which language learning and processing is constrained by processes operating at divergent structural levels in divergent time/process frames (MacWhinney, 2015a). To account for patterns of disfluency, stuttering, code-switching, and conversational sequencing, we look at the constraints imposed by online lexical access, patterns of cortical activation, and social affiliation. Some of these constraints operate across a timeframe of milliseconds and others, like social affiliation, extend across decades.
    \item These cases of complete language attrition contrasts with the minimal level of language attrition experienced by older immigrants (MacWhinney, 2018).[50] It is difficult to account for this in terms of entrenchment alone, because the young learners have used their L1 continually for as much as 6 years. Instead one needs to look at the role of social support for young immigrants, pressures to adopt L2, and the role of literacy in helping older learners consolidate their access to L1 forms
    \item The application of the model to child language acquisition focuses on the role that cue availability and reliability play in determining the order of acquisition of grammatical structures. The basic finding is that children first learn the most available cue(s) in their language. \textbf{If the most available cue is not also the most reliable, then children slowly shift from depending on the available cue to depending on the more reliable cue}
    \item The Competition Model emphasises that processing occurs through \textbf{pattterns of connectivity and activation rather than serial processing}, and is an empiricist model, meaning that it emphasises \textbf{the role of input in modifying the weights of these connections.} 
    \item The ideas of ‘cue validity’ and ‘cue strength’, (i.e. \textbf{how predictably} a certain syntactic, semantic or phonological property of a word can deliver particular information about surrounding words and grammaticality) \textbf{represent the information which is learnt from exposure to input during the process of language-learning.}
    \item In order to validate the Competition Model for SLA, we would expect to see evidence that any aspects of the L1 which can possibly be transferred to the L2 are \textbf{directly transferred at the start of the language-learning process,} including lexical analogues and all grammatical cues (for instance, if word order is a strong cue in a learner’s L1, they will weight it equally strongly in their L2 as a beginner regardless of its cue strength in the L2). This is because the Competition Model suggests that \textbf{all linguistic and non-linguistic processing uses the same set of cognitive structures,} and so these cognitive structures will \textbf{generalise patterns wherever possible} – including from an L1 to an L2. 
    \item According to this model, SLA is a process of reconfiguring the weights connecting functions to clusters of forms, and so we would expect to see a gradual \textbf{rebalancing of cue strength towards L2-appropriate distributions,} as well as a gradual decrease in the use of structures directly imported from L1, over the course of SLA.
    \item We do see evidence in the data of direct transfer of grammatical constructions from L1 to L2 amongst the Korean L1 speakers. Korean does not have articles and does not decline its demonstrative pronouns, and so omitting D-elements (especially articles) entirely or neglecting to produce them in plural contexts is \textbf{evidence of direct transfer from Korean} – particularly when we consider that Leonie had no trouble producing them, and so they do not pose any inherent additional processing difficulty. Lexical marking of plurals with both numerals and quantifiers, however, is the primary method of pluralisation in Korean, and so this is transferred directly into L2 German despite being secondary to morphological marking in German (as shown by the much higher prevalence of morphological marking in Leonie’s plurals). In addition, L1 Korean speakers may have less difficulty with N-plural markers than with D-elements (especially articles) due to the fact that there is a plural inflectional morpheme in Korean, albeit one which is mainly reserved for emphatic use. The Competition Model would suggest that \textbf{it is easier to pick up an L2 feature when there is an L1 analogue} because mapping an L2 form to an existing L1 meaning is easier than acquiring a brand new form with a brand new associated meaning; this seems to explain why L1 Korean speakers have more trouble acquiring articles than acquiring plural suffixes, since the latter exist in Korean and the former do not.
    \item An additional piece of evidence which would support the Competition Model, but is not present here, is evidence that \textbf{overall language ability is positively correlated with the use of L2-like cues to determine grammaticality.} Since grammaticality judgement was not assessed here, we cannot gain a sense of the factors determining cue strength in L1 and L2 learners, so a key prediction of the model is missing.
    \item In short: as well as accounting for competition (whose results are determined by cue validity and cue strength) it is necessary to account for structural analysis, i.e. the extraction of syntactic and semantic information from perceptual realities and the statistical learning process behind item-based learning, as well as time/process constraints i.e. constraints imposed by neural processing and social factors.
    \subsection{Questions to fit the model to data}
    \item Are children learning the most available and/or reliable cues in the language and basing inferences on that? For SLA, are they directly transferring information about cues from their first language before they pick up the pattern?
    \item Is there obvious competition between possible target items?
    \item Are children memorising combinations of items, e.g. possessive pronoun followed by noun, and generalising from these?
    \item Is it provable that the brain is/neural mechanisms are actively involved in perspective-shifting while a child is processing a concept like agency (or any theta-role)?
    \item Is processing non-serial/non-rule-based?
    \item Is processing primarily input-based and functional (i.e. does a generally emergentist account make sense here?)?
    \item Is SLA a gradual rebalancing of cues?
    \item Is it useful to suggest that linguistic and nonlinguistic processing are using the same neural mechanisms? Is linguistic pattern recognition equivalent to nonlinguistic pattern recognition?
    \subsection{Fundamental Difference Hypothesis}
    \item Foreign language learning contrasts with native language development in two key respects: It is unreliable and it is nonconvergent. At the same time, it is clear that foreign languages are languages. The fundamental difference hypothesis (FDH) was introduced as a way to account for the general characteristics of foreign language learning. The FDH was originally formulated in the context of the theory of rich Universal Grammar, and this theory has guided much foreign language acquisition research over the past two decades. However, advances in the understanding of language have undermined much of the supporting framework. The FDH—indeed all of SLA research—must be rethought in light of these advances. It is proposed here that (a) foreign language grammars make central use of patches, which are also seen as peripheral phenomena in native languages; (b) non-domain-specific processes are used in foreign language acquisition, but that these are also employed—although more effectively because they are integrated into the language system—by native language development; and (c) foreign language online processing relies heavily on the use of shallow parses, but these are also available in native language processing, although less crucially. (PsycINFO Database Record (c) 2016 APA, all rights reserved)
\end{itemize}


\end{document}